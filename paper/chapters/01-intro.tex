\chapter{Introduction: A New Paradigm for Machine Pedagogy}

\section{The Imitation Ceiling}
The modern era of Artificial Intelligence is largely defined by the success of imitation learning. However, this success masks a fundamental limitation: an imitative system is a high-fidelity mirror, but a mirror cannot create a new image. In education, this translates to an \textbf{Imitation Efficacy Ceiling}. An AI tutor trained on a dataset of human teaching examples can learn to be as effective as the best teacher in that dataset, but it can never systematically surpass them. This thesis argues that to create truly intelligent pedagogical agents, we must move beyond imitation.

\section{Thesis Statement: Pedagogy as a Self-Discovering Game}
This dissertation introduces and validates a new paradigm for artificial intelligence: \textbf{Pedagogy as a Self-Discovering Game}. We posit that inventive, effective, and robust pedagogical strategies can emerge from the co-evolutionary dynamics of autonomous agents operating within a principled, game-theoretic framework. We position this work alongside foundational shifts in AI research, such as Self-Play in Reinforcement Learning \citep{silver2017mastering}. Just as self-play unlocked superhuman strategic gameplay, we propose that a principled "self-teaching" game can unlock the discovery of superhuman pedagogy.

\section{Core Contributions}
This thesis will make four primary contributions, composed of a rigorous theoretical framework and a series of monumental, yet computationally feasible, experiments designed to validate it.

\begin{enumerate}
    \item \textbf{A New Theoretical Quintet for Emergent Pedagogy:} We provide a chain of five theorems that establish the foundations of our paradigm: an impossibility theorem for imitation, a possibility theorem for discovery, a phase transition theorem for invention, a robustness theorem for alignment, and a necessity theorem proving no free lunch.
    \item \textbf{The Self-Structuring Cognitive Agent (SSCA):} We design and implement a novel agent architecture that serves as a computational model of human pedagogy, learning not only a teaching policy but also simultaneously building an internal, dynamic "world model" of the conceptual space it is teaching.
    \item \textbf{The First Empirical Demonstration of Breaking the Imitation Ceiling:} We will conduct a series of experiments, feasible on a free-tier cloud budget, that provide the first clear, statistically significant evidence of an AI system discovering pedagogical strategies superior to those in its initial expert dataset.
    \item \textbf{A Tightly-Coupled Theoretical-Empirical Loop:} Each experiment is explicitly designed as the empirical analogue of a core theorem, creating a closed, unassailable argument that bridges formal theory and empirical validation.
\end{enumerate}