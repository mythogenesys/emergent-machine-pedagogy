\noindent The dominant paradigm for AI tutoring is imitation learning. While successful, this approach is fundamentally constrained by a theoretical ceiling on its inventive capacity. This thesis proposes a new paradigm: \textbf{Pedagogy as a Self-Discovering Game}. We argue that inventive teaching strategies can emerge from the interactions of autonomous agents within a principled, game-theoretic framework, analogous to how self-play unlocked superhuman strategies in games.

We formalize this paradigm by introducing the COGNITA stochastic game. Our theoretical contributions form a comprehensive framework that establishes this as a new, robust field of inquiry. We prove the \textbf{Imitation Efficacy Ceiling}, an impossibility theorem on imitation. We then prove a \textbf{Discovery-Efficacy Tradeoff Theorem}, establishing the mathematical license for our system to invent novel strategies that exceed this ceiling, connecting it to principles of information bottleneck theory. We theorize a \textbf{Critical Diversity Threshold}, a phase transition where emergent curricula appear, linking our AI system to established models in cognitive science from Piaget and Vygotsky. We provide a \textbf{PAC-Verifier Guarantee} that serves as a formal alignment guarantee for the system's safety and reliability. Finally, we prove a \textbf{No Free Lunch for Pedagogy Theorem}, showing that invention is provably impossible without the core components of our system.

To validate this theory, we propose a series of computationally feasible experiments. Using a custom-built Self-Structuring Cognitive Agent (SSCA) as a computational model of pedagogy, we will provide the first empirical evidence of a system breaking the imitation ceiling. Each experiment is designed as the explicit empirical analogue of a corresponding theorem. This work aims to shift the frontier of AI research from building systems that retrieve knowledge to creating systems that can autonomously discover and structure the principles of pedagogy, with profound implications for cognitive science, education, and AI safety.